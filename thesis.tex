\documentclass[hyperref,german,diplominf]{cgvpub}

\author{Maximilian Richter}
\title{Validierung einer Gathering-Strategie zur Szenendiskretisierung f\"ur Partikeldaten im Kontext der Berechnung von ambienter Verdeckung}
\birthday{8. April 1993}
\placeofbirth{Schleiz}
\matno{3802290}
\betreuer{Dipl.-Medieninf. Joachim Staib}
\bibfiles{literatur}
%\problem{Text der Aufgabenstellung...}
%\copyrighterklaerung{}
%\acknowledgments{}
% \abstracten{abstract text english}
% \abstractde{ Zusammenfassung Text Deutsch}

\begin{document}

\chapter{Introduction}

\section{Motivation}
Global illumination allows to visualize complex structures in particle data, that would otherwise remain unseen. Ambient occlusion mimics indirect diffuse light by calculating the level of occlusion for visible surfaces. The calculation of the ambient terms is costly, but several methods exist to approximate it. One of them, Voxel Cone Tracing, uses an discretized voxel representation of the geometry. This requires a method to transfer particles into a voxel grid. Scattering iterates all particles and adds their partial contributions to the intersected voxels. Parallel scattering processes the particles simultaneously, leading to race conditions whenever multiple spheres intersect the same voxel. Synchronization has to assure correct summation of the contributions. As synchronization increases, parallelism decreases. Therefore scattering performs the worse, the more dense the particle data is.

\section{Task}
To find a particle voxelization technique that better fits the parallel nature of graphics processing units a gathering approach is validated. Gathering determines for each voxel which particles it intersects with. For efficient particle search around voxels an acceleration data structure is required. Starting from an unordered list of particles an algorithm is designed to first build a hashed uniform grid and then use it to calculate the voxel densities. To compare the gathering approach against scattering the construction time is measured for multiple particle clouds of variable density. The created voxel data structure is used to calculate ambient occlusion using voxel cone tracing. To evaluate the quality of the proposed solution a ground truth renderer using ray-casting is implemented.

\section{Outline}

The work is organized as follows. In chapter 2 related works are discussed. First concerning acceleration data structures on particles, followed by voxelization techniques and finally ambient occlusion for particle data.

Chapter 3 presents the proposed gathering based voxelization technique. 

Chapter 4 details the techniques used to calculate ambient occlusion.

In chapter 5 implementation details are given.

Chapter 6 finally presents the results of the work.



\chapter{Related Work}

\chapter{Gathering Voxelization}

\chapter{Ambient Occlusion}

\section{Voxel Cone Tracing}
cone tracing idea + cone approximation with voxels
construction of hierarchy

\section{Ray Casting}
idea + hemisphere sampling
ray sphere intersection math


\chapter{Implementation}

\section{Gathering Voxelization}

\section{Ambient Occlusion}

\section{Voxel Cone Tracing}
-> hemisphere of n cones per frame
-> variable cone size + variable voxel grid size for comparison

\section{Ray Casting}
-> GPU Pathtracer on spheres
-> progressive pipeline
-> also on voxels for comparison

\chapter{Results}
\section{Evaluation}

1. Qualitativ

VCT on standard resolution grid - ray tracing on particle data -> Fehlerquellen:

VCT on high resolution grid -> voxelization error

Ray casting/Fine cones on voxel grid -> cone tracing error

2. Quantitativ

Scattering vs Gathering Performance + Speicherverbrauch

\section{Discussion}
\chapter{Conclusion}


\end{document}